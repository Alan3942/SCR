\documentclass{article}

\title{Errata and Notes for  \\
\emph{\sc{Statistical Computing with R}}\\
First Edition\\
Chapman \& Hall/CRC (2008)}
\author{Maria L. Rizzo}
\date{28 October 2017}  

\usepackage{amsmath}

% Set up the images/graphics package
\usepackage{graphicx}
\setkeys{Gin}{width=\linewidth,totalheight=\textheight,keepaspectratio}
\graphicspath{{graphics/}}

% The fancyvrb package lets us customize the formatting of verbatim
% environments.  We use a slightly smaller font.
\usepackage{fancyvrb}
\fvset{fontsize=\normalsize}

\begin{document}

\maketitle% 

\begin{abstract}
\noindent This document contains errata, notes, and some programming notes
for the first edition (2008) of "Statistical Computing with R", ISBN 978-1-58488-545-0
(Hardcover). 
\end{abstract}


\section{Errata}

\begin{enumerate}
\item[p. 11] \emph{Section 1.6, Data Frames.} 
For R versions $>=$ 3.0.0, \texttt{mean} for data frames is defunct.
Change the function \texttt{mean} to \texttt{colMeans}:
\begin{verbatim}
by(iris[,1:4], iris$Species, colMeans)
\end{verbatim}

\item[p. 30]Last line:\emph{Section 2.3, Chisquare and $t$.} 
$x \geq 0, \nu=1,2,\dots$.


\item[p. 35]\emph{Section 2.5.} In the first two displayed equations: $\lim\limits_{n
\to \infty}$.


\item[p. 58] Item 4. \emph{Section 3.4}.
\begin{align*}
Z_1 &= \sqrt{-2 \log U} \cos(2 \pi V),\\
Z_2 &= \sqrt{-2 \log U} \sin(2 \pi V)
\end{align*}

%% Example 4.10
%% lattice::parallel is deprecated
%% use parallelplot instead

\item[p. 114]Example 4.11 \emph{Section 4.5.3}, code for segment plot in Figure 4.10. 
In \texttt{stars}, the labels argument must be a vector of character strings: use \texttt{levels} of the factor for labels.
\begin{verbatim}
stars(x[4:8], draw.segments = TRUE,
  labels = levels(x$sp), nrow = 4,
  ylim = c(-2,10), key.loc = c(3,-1))
\end{verbatim}

\item[p. 121] \emph{Section 5.2.1}, Example 5.2.
In step 2: $\overline{g(X)}=\frac 1m \sum_{i=1}^m g(X_i)$.

\item[p. 132]\emph{Section 5.5.} 
In sentence above (5.9): $\hat \theta_c=g(X)+c(f(X)-\mu)$.

\item[p. 142]\emph{Section 5.6}, Example 5.10. \\
The standard errors will be the vector
\verb|se/sqrt{m}\verb|, so
the summary is:
\begin{Verbatim}[fontsize=\small]
> rbind(theta.hat, se/sqrt(m))
                 [,1]        [,2]      [,3]         [,4]        [,5]
theta.hat 0.524114007 0.531358351 0.5461507 0.5250698759 0.526049238
          0.002436559 0.004181264 0.0096613 0.0009658794 0.001427685
\end{Verbatim}

\item[p. 143]In \emph{Section 5.6} ``Variance in Importance Sampling'',
in the equation for $Var(\hat \theta)$, $ds$ should be $dx$ in the integral.



\item[p. 158]Table 6.1 \emph{Section 6.2.2}. The right two columns
reporting $n \, \widehat{se}$
are not correct. To obtain results for $n \, \widehat{MSE}$ in Table 6.1,
\texttt{set.seed(522)}. The corrected table is given below. \medskip



\begin{tabular}{rrrrrrr}
  \hline
&  \multicolumn{2}{c}{Normal} & \multicolumn{2}{c}{$p=0.95$} &
   \multicolumn{2}{c}{$p=0.90$} \\
   \hline
k & $n\, \widehat{MSE}$  & $n\, \widehat{se}$ & $n\, \widehat{MSE}$
&  $n\, \widehat{se}$ & $n\,
\widehat{MSE}$ &  $n\, \widehat{se}$ $\vphantom{{\sum\limits^M}}$\\
  \hline
  1 & 0.976 & 0.140 & 6.229 & 0.353 & 11.485 & 0.479 \\
  2 & 1.019 & 0.143 & 1.954 & 0.198 & 4.126 & 0.287 \\
  3 & 1.009 & 0.142 & 1.304 & 0.161 & 1.956 & 0.198 \\
  4 & 1.081 & 0.147 & 1.168 & 0.153 & 1.578 & 0.178 \\
  5 & 1.048 & 0.145 & 1.280 & 0.160 & 1.453 & 0.170 \\
  6 & 1.103 & 0.149 & 1.395 & 0.167 & 1.423 & 0.169 \\
  7 & 1.316 & 0.162 & 1.349 & 0.164 & 1.574 & 0.177 \\
  8 & 1.377 & 0.166 & 1.503 & 0.173 & 1.734 & 0.186 \\
  9 & 1.382 & 0.166 & 1.525 & 0.175 & 1.694 & 0.184 \\
  10 & 1.491 & 0.172 & 1.646 & 0.181 & 1.843 & 0.192 \\
   \hline
\end{tabular}
\bigskip


 \item[p. 187]Line 1. \emph{Section 7.1.1}, Example 7.2, paragraph following the code.
 Remove hat from the first `se'.

\item[p. 200]Example 7.10. \emph{Section 7.4.3}, above R note 7.3.

Misplaced right paren.; correction:
\begin{verbatim}
    #normal
    print(boot.obj$t0 + qnorm(alpha) * sd(boot.obj$t))
\end{verbatim}

\item[p. 204]\emph{Section 7.5}. 
$$
 \hat a = \frac{\sum_{i=1}^n (\overline{ \theta_{(.)}} -
 \theta_{(i)})^3}
 {6 \left(\sum_{i=1}^n (\overline{\theta_{(.)}} -
 \theta_{(i)})^2\right)^{3/2}},
 \eqno(7.11)
$$

\item[p. 240]Example 8.13 (Distance covariance test): In the function \texttt{ndCov2}, line 3 should be
\begin{verbatim}
    q1 <- p + 1
\end{verbatim}

\item[p. 246]First paragraph, in second sentence, \emph{Section 9.1.2}.
``$n$ tends to infinity'' should be ``$m$ tends to infinity''.

 \item[p. 260]\emph{Section 9.2.4}, Example 9.6. 
 Last paragraph: delete the second sentence ``Then an observed sample is generated.''


\item[p. 263]Example 9.7. \emph{Section 9.3}.
$$
  E[X_2|x_1] = \mu_2 + \rho\frac{\sigma_2}{\sigma_1}(x_1-\mu_1)
$$

\item[p. 283]\emph{Section 10.1.1}. 
Sturges's Rule: ``For large $k$ (large $n$) the distribution
of Binomial($n, 1/2)$ is \dots''

\item[p. 287]\emph{Section 10.1.1}; Scott's Normal Reference Rule. 
In equation (10.3) for $h_n^*$: replace $n$ with $n^{-1}$. See
Scott [241] equation (3.15).


\item[p. 312]Example 10.15. \emph{Section10.3.3}. 
The mean vectors used to generate the samples in the code
and in the plots in Figure 10.13 are
$$
 \mu_1= \left[
         \begin{array}{c}
           0 \\
           0 \\
         \end{array}
       \right], \quad
 \mu_2= \left[
         \begin{array}{c}
           1 \\
           3 \\
         \end{array}
       \right], \quad
 \mu_3= \left[
         \begin{array}{c}
           4 \\
           -1 \\
         \end{array}
       \right].
 $$


 \item[p. 315]Exercise 10.5. The skewness adjustment factor is
 given in (10.8).

\end{enumerate}

\section{Notes}



\begin{enumerate}

 \item[p. 56] \emph{Section 3.3}.  Remark on Example 3.7. Although 6 is an upper bound, it
 is not the least upper bound. The generator is more efficient if
 $c=1.5$, the maximum value of $f(x)/g(x)$ for $0 \leq x \leq 1$.


 \item[p. 57] \emph{Section 3.3, Example 3.7}, code above para. 2:\\
  \texttt{\#See Ch. 2} 
 

 \item[p. 71--72]Example 3.16, summary statistics. \emph{Section 3.6.1}.
 The \texttt{rmvn.eigen} generator takes the
 covariance matrix \texttt{Sigma} as an argument, so in general one may
 want to display \texttt{cov(X)} for comparison with \texttt{Sigma} rather than the sample
 correlation matrix \texttt{cor(X)}. (Here our \texttt{Sigma}
 was a correlation matrix.)



 \item[p. 178--179]Examples 6.14--6.15.  \emph{Section 6.4}. 
 Although mathematically it is not
 an error, it is unnecessary to subtract the sample means in the
 expression \texttt{tests} of Example 6.14 because the sample
 means are subtracted in the function \texttt{count5test}.
 Mathematically, if $Z_i=X_i-\overline X$, then $\overline Z=0$.
 The same is true in the expression for \texttt{alphahat} in
 Example 6.15. The \texttt{count5test} can be
 applied without centering the data first, as in Example 6.16.


\item[p. 225--228]Examples 8.4-8.6.  \emph{Section 8.3, Nearest Neighbor Tests}.  
    The \texttt{knnFinder} package with \texttt{nn} function for finding nearest neighbors has been withdrawn from CRAN. These examples have been revised using the \texttt{ann} function in the \texttt{yaImpute} package.


\item[p. 323-324]Example 11.3.  \emph{Section 11.1, Evaluating Functions}.
In \texttt{system.time} the timings are hardware dependent; however, the vectorized version should be faster on all platforms.


 \item[pp. 338--339]Example 11.11.  \emph{Section 11.5}.
 The first code snippet will produce a graph similar to Figure 11.3 but with $x$-axis ranging from about 2 to 8. To produce Figure 11.3 as shown, replace 8 with 15 in \texttt{seq(2, 8, .001)}.

\item[p. 341]Example 11.12.  \emph{Section 11.5}.
The histograms will of course vary slightly from Figure 11.4 because the data is generated at random. According to my notes, \texttt{set.seed(333)} before the first line of code should produce samples matching the histograms as shown on page 341. See the note below concerning \texttt{par(ask=TRUE)} to wait for user input before displaying each graph.


\item[p. 342--343]Example 11.13.  \emph{Section 11.6}.\\
\texttt{set.seed(333)} was set prior to run.


\end{enumerate}


\section{Programming Notes}

\begin{enumerate}
 \item \texttt{curve} is convenient for plotting a function. It can replace \texttt{lines} in some examples; e.g. in Example 3.2 to add the density curve to the histogram, instead of \texttt{lines}
     we can use:

     \verb|curve(3*x^2, add=TRUE)|

 \item Displaying a sequence of graphs: \texttt{par(ask=TRUE)} has the effect that the user is asked for input before each new figure is drawn. Follow it with \texttt{par(ask=FALSE)} to restore to default behavior.

\item \texttt{sapply} can be used instead of \texttt{apply} in some examples, which eliminates the need for MARGIN and the need for
    the argument to have a dimension attribute.
    See e.g. Example 5.4 on page 123. The lines:

\begin{verbatim}
  dim(x) <- length(x)
  p <- apply(x, MARGIN=1, function(x, z) {mean(z < x)}, z=z)
\end{verbatim}
    can be replaced with either version below:
\begin{verbatim}
  p <- sapply(x, FUN=function(x,z) mean(z<x), z=z)
  p <- sapply(x, function(x) mean(z<x))
\end{verbatim}

\item A more elegant approach to the comparison of generators in Example 3.19
is to wrap the repeated statements in a function that takes the
name of the generator (e.g. rmvn.eigen) as an argument. An example
of a function that has a functional argument is \texttt{boot}; see
Example 7.10 for a typical example.

\end{enumerate}

\paragraph{Acknowledgements}

Thanks to several readers who sent comments, corrections, and suggestions.



\end{document}
